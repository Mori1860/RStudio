\documentclass[12pt]{article}
\usepackage[utf8]{inputenc} %for special characters (now often redundant in modern LaTeX engines)
\usepackage{graphicx}
\usepackage{float}     %control over floating objects (e.g., tables, figures) with the [H] placement option.
\usepackage{fancyhdr}    %Customizes headers and footers (e.g., page numbers,..
\usepackage{lastpage}      %reference the total page count
\usepackage{hyperref}     %Adds hyperlinks (e.g., for citations, URLs, 
\usepackage{booktabs}    %for table formatting
\usepackage{textcomp}    % Adds extra text symbols (e.g., currency,..
\usepackage{tabularx}   % adjustable-width tables



\title{Data Management and Visualization Exam}
\author{Morteza Bakhtiari}
\date{\today}  %automatic date

\pagestyle{fancy}
\fancyhf{}
\lhead{\author}
\chead{Data Management and Visualization Course}
\rhead{\today}
\lfoot{Page \thepage\ of \pageref{LastPage}}
\rfoot{Data Management Exam}



% Begin document content
\begin{document}

\maketitle

\tableofcontents
\listoftables
\listoffigures


% Section with label for referencing
\section{Exercise I: Document Setup} \label{sec:setup}

% Section content begins
This section shows the setup for the content of the exam problem set

% Itemized list
\begin{itemize}
    \item A table of contents, list of tables, and list of figures
    \item Proper section organization
    \item Descriptive text explaining the setup process
\end{itemize}

% Description of required packages  teletype font 
The setup required several LaTeX packages:
\begin{itemize}
    % Package for headers/footers, text teletype
    %Displays text in a fixed-width/monospaced font (like code or terminal output), Similar to how text appears in code editors or command line interfaces
    \item \texttt{fancyhdr} for custom headers and footers
    % Package for page numbering
    \item \texttt{lastpage} for page numbering
    % Package for graphics
    \item \texttt{graphicx} for figure inclusion
    % Package for hyperlinks
    \item \texttt{hyperref} for better PDF navigation
\end{itemize}

% Second section with unique identifier ref
\section{Exercise II: Data Preparation and Cleaning} \label{sec:analysis}
\subsection{Data Summary}
% Section description
This section contain the data preparation with tables and figures. This dataset includes 1044950 observations of  83 variables. \\ Key Variables:

\begin{itemize}
    \item \textbf{Demographics:} 
        \begin{itemize}
            \item personid: Unique identifier (character)
            \item year: Ranges from 1993 to 2013 (median = 2003)
            \item  age: 23–99 years (mean = 45.1, but has 52,277 NA values)
            \item gender, raceth, bthus (Place of birth): Categorical

        \end{itemize}
            
    \item \textbf{Education: }  
        \begin{itemize}
            \item dgrdg (Type of highest certificate or degree)
            \item hd03y5 (Year of highest degree 2003-onward (5 year intervals))
            \item ndgmed (Field of major for highest degree): Categorical
        \end{itemize}

           \item \textbf{Employment:} 
                \begin{itemize}
                    \item hrswkgr: (hours per week typically worked (group)).
                    \item salary: Numeric (salary has extreme values: median = %$68k)
                    \item jobsatis, satsal (job/salary satisfaction): Categorical
                \end{itemize}
                
            \item \textbf{Job Characteristics}
                \begin{itemize}
                    \item wamgmt (Work activities on principal job: management and administration)
                    \item waprod (Work activities on principal job: production, operations, maintenance)
                    \item watea (Work activities on principal job: teaching): Categorical. 
                \end{itemize}
                
            \item \textbf{Notable Observations:}
                \begin{itemize}
                    \item Missing Data:
                        \begin{itemize}
                            \item age has 52,277 NAs
                            \item salary has 52,297 NAs 
                        \end{itemize}
                    \item Time Coverage: Data spans 1993–2013 (likely longitudinal/panel data).
                    \item Data class: Categorical Dominance: Most variables are character type (may need conversion to factors for analysis).
                
                \end{itemize}  

\begin{table}[ht]
\centering
\caption{Top 10 Missisng summary data} 
\label{tab:missing_summary}
\begin{tabular}{|l|r|r|r|}
  \toprule
 & variable & missing\_count & missing\_pct \\ 
  \midrule
1 & satresp & 737767.00 & 77.66 \\           %job's level of responsibility
  2 & satadv & 737725.00 & 77.66 \\ 
  3 & satsoc & 737722.00 & 77.66 \\           % job's contribution to society
  4 & satloc & 737687.00 & 77.66 \\ 
  5 & satchal & 737649.00 & 77.65 \\ 
  6 & satben & 737636.00 & 77.65 \\ 
  7 & satsal & 737633.00 & 77.65 \\ 
  8 & satind & 737569.00 & 77.64 \\ 
  9 & satsec & 737405.00 & 77.63 \\ 
  10 & jobins & 642778.00 & 67.66 \\ 
   \bottomrule
\end{tabular}
\end{table}

            
\end{itemize}
\subsection{Data problems and solving}
\begin{itemize}
    \item Every person should only be surveyed once per year. there are 94,996 observations duplicated which have to be solved by distinct command.
    \item Variables are widely coded inconsistently. There are 52,277 missing values in Age variable, 52,297 values in Salary variable. salary has extreme values: median = 68k , but max = 9,999,999, likely censored)

    \item There are some typo errors in gender variable detailed in below table which caused inconsistency in the variable. by recoding the gender variable the problem has been solved.
    
    % Begin table environment with H positioning
\begin{table}[H]
    % Center table
    \centering
    % Table caption
       % Table label for referencing
    \label{tab:gender_dist}
    % Begin tabular environment with 3 centered columns and lines
    \begin{tabular}{|c|c|c|c|c|}
        % Horizontal line
        \hline
        % Table headers
         & Female & Female & Male & Male\\ \hline
        % First row of data
        Initial number & 260820 &  65415 & 460581 & 115458 \\ \hline
        % Second row of data
        Recoded error-free number & 326235 & 0 & 576039 & 0 \\ \hline
    % End tabular environment
    \end{tabular}
     \caption{Gender distribution Table}
\end{table}

    \item The age variable consists of values between 23 and 99. Because the distance between the last age and the one before reveals a huge gap (24 years) and this category includes only 6 values, it could be considered as a potential missing values in between.

\begin{figure}[h]
\centering
\includegraphics[width=0.8\textwidth]{Age_dist.png}
\caption{Age Distribution}
\end{figure}    
    
    \item There are some typo errors in dgrdg(highest degree) variable detailed in below table which caused inconsistency in the variable. by recoding the gender variable the problem has been solved.

%%%%%%%%%%%%%%%%%%%%%%%     Read %%%%%%%%%%%%%
\begin{table}[h]
    \begin{flushright}  % Right-aligns the entire table
    \renewcommand{\arraystretch}{1.2} % Improves row spacing
    \setlength{\tabcolsep}{8pt} % Adjusts column padding
    \begin{tabular}{|l|r|r|r|r|r|r|r|r|}
        \hline
        & \multicolumn{2}{c|}{Bachelor} & \multicolumn{2}{c|}{Doctorate} & \multicolumn{2}{c|}{Master} & \multicolumn{2}{c|}{Professional} \\
        \cline{2-9}
        & Initial & Recoded & Initial & Recoded & Initial & Recoded & Initial & Recoded \\
        \hline
        Count & 280,112 & 350,788 & 237,134 & 296,088 & 174,098 & 217,423 & 30,341 & 37,811 \\
        \hline
        Other & 70,676 & 0 & 58,954 & 0 & 43,325 & 0 & 7,470 & 0 \\
        \hline
    \end{tabular}
    \caption{Degree Distribution with Initial and Recoded Values}
    \label{tab:degree_dist}
    \end{flushright}
\end{table}

    \item There are also some typo errors in "ctzusin" variable which was recoded and solved. The type of typo errors categorized into 3 groups which can be seen in table below.
\begin{table}[h]
    \begin{flushright}  % Right-aligns the entire table
    \renewcommand{\arraystretch}{1.2} % Improves row spacing
    \setlength{\tabcolsep}{8pt} % Adjusts column padding
    \begin{tabular}{|l|r|r|r|r|}
        \hline
        & \multicolumn{2}{c|}{No} & \multicolumn{2}{c|}{Yes} \\
        \cline{2-5}
        & Initial & Recoded & Initial & Recoded  \\
        \hline
        Category 1 & 5,332 & 71811 & 62,384 & 830,463 \\
        \hline
        Category 2 & 61,106 & 0 & 705,479 & 0 \\
        \hline
        Category 3 & 5,373 & 0 & 62,600 & 0 \\
        \hline
        \end{tabular}
    \caption{US citizen Distribution (Initial and Recoded)}
    \label{tab:degree_dist}
    \end{flushright}
\end{table}

    \item Under the "lfstat" (Employment status) there are several typo errors causing repeated errors(Employed, Unemployed and Not in the labor force) which was separated into two categories.

\begin{table}[h]
    \begin{flushright}  % Right-aligns the entire table
    \renewcommand{\arraystretch}{1.2} % Improves row spacing
    \setlength{\tabcolsep}{8pt} % Adjusts column padding
    \begin{tabular}{|l|r|r|r|r|r|r|}
        \hline
        & \multicolumn{2}{c|}{Employed} & \multicolumn{2}{c|}{Not in the labor force} & \multicolumn{2}{c|}{Unemployed} \\
        \cline{2-6}
        & Initial & Recoded & Initial & Recoded & Initial & Recoded \\
        \hline
        Count & 620,086 & 774,830 & 83,722 & 104,823 & 17,997 & 22,526  \\
        \hline
        Other & 154,744 & 0 & 21,101 & 0 & 4,529 & 0 \\
        \hline
    \end{tabular}
    \caption{Employment Distribution with Initial and Recoded Values}
    \label{tab:degree_dist}
    \end{flushright}
\end{table}
    \item The categorical data is recommended to be presented in Factor format. Therefore 
\end{itemize}





\section{Exercise III: Data Aggregation and Descriptive Statistics}
In this section, we want to analyze job satisfaction patterns by educational attainment. Create a comprehensive table showing 2-3 different job satisfaction measures (e.g., overall job satisfaction, salary satisfaction, or advancement opportunities) by degree type and field of study.

\subsection{Analyze job satisfaction patterns by educational attainment} 
\subsubsection{creating job satisfaction dataset}
In this part, we generate a specific dataset including person identifiers, Degrees and satisfaction measurement data for analyzing purpose.

\subsubsection{cleaning dataset}
In this part, we need to clean the data in terms of missing values, typo errors, duplicated values and any other misleading information. Because some values are in the same categories but with different lower/upper case characters, we have to standardize them and cut out the tailing and whitespaces first. Then we can filter to keep rows with AT LEAST ONE non-NA satisfaction measure.

\subsubsection{Analyzing the satisfaction by educational status}
In this section, we use the cleaned data to analyze the satisfaction variables(Job Satisfaction, Salary Satisfaction and Opportunity Satisfaction) by the Educational Status. We can evaluate conditions sequentially and assign 4(very satisfied) to 1(very dissatisfied). Then calculate the percentage of Salary Satisfaction and Advancement Opportunity in that regards.  

% latex table generated in R 4.4.2 by xtable 1.8-4 package
\begin{table}[H]
\centering
\begin{tabular}{|l|r|r|r|r|r|}
  \hline
  \toprule
Degree & Job Satisfaction & Salary Satisfaction &  Advancement Opp & Nunmber \\ 
  \hline
  \midrule
Professional & 3.51 & 30.55 & 30.46 & 22835 \\ 
  \hline
Doctorate  & 3.41 & 32.67 & 32.63 & 161828 \\
  \hline
Master's & 3.36 & 38.02 & 37.92 & 124131 \\ 
  \hline
Bachelor's & 3.30 & 33.53 & 33.57 & 186879 \\ 
  \hline
  \bottomrule
\end{tabular}
\caption{Job Satisfaction Summary by Degree} 
\label{tab:satisfaction}
\end{table}      

    A
In the table above you can see 
\begin{itemize}
    \item \textbf{The average of Job Satisfaction Scores (1-4 Scale)} \\
Professional degree holders report the highest job satisfaction (3.51), followed by Doctorate (3.41), Master’s (3.36), and Bachelor’s (3.30).
    \item The scores are relatively close, suggesting degree type has a modest (but not drastic) impact on overall satisfaction.
    \item\textbf{Salary Satisfaction percentage}  \\
    Master’s degree holders are the most satisfied with their salary (38.02), while Professional degree holders are the least satisfied (30.55).
    \item Surprisingly, higher degrees (Doctorate/Professional) do not correlate with higher salary satisfaction.
    \item \textbf{Advancement Opportunities percentage} \\
    Mirroring salary trends, Master’s holders again report the highest satisfaction with advancement opportunities (37.92)
    \item The gap between degrees is small (30–38), suggesting advancement satisfaction is broadly similar across education levels.
    \item \textbf{Sample Size ("Opp" Column)} \\
    Bachelor’s degree holders dominate the sample (186,879 responses), followed by Doctorate (161,828). The "Professional" category has the smallest sample (22,835), which may affect the reliability of its metrics.
    \item \textbf{Missing Data} \\
    The "Number" column is empty, which might indicate incomplete data or a placeholder for additional metrics (e.g., response rates).
 
\end{itemize}

\subsection{Examining career-education alignment and its relationship to satisfaction. }
\subsubsection{Degree alignment with satisfaction}
Stronger alignment between degree and job correlates with higher satisfaction across all measured factors.
\begin{itemize}
    \item \textbf{Job Satisfaction (1-4 Scale)}
    Highest for closely related jobs (3.50), decreasing for somewhat related (3.30) and not related (3.10).
    \item \textbf{Salary \& Advancement Satisfaction} 
    Both metrics follow the same trend:
        Closely related: ~34.8\% satisfied with salary, ~34.7\% with advancement.
        Not related: ~32.5-32.6\% satisfied (2-3\% lower).
    \item \textbf{Sample Size (N)}
    Majority of respondents work in closely related jobs (289,856 vs. 61,572 "not related").
\end{itemize}


\begin{table}[ht]
\centering
\begin{tabular}{|r|r|r|r|r|}
  \hline
Job related degree & Job Satisfact(1-4) & \%Satisfy Salary & \%Satisfy by Advancement & N \\ 
  \hline
closely related & 3.50 & 34.80 & 34.70 & 289856\\ \hline
  somewhat related & 3.30 & 33.90 & 33.90 & 119480 \\ \hline
  not related & 3.10 & 32.50 & 32.60 & 61572 \\ 
   \hline
\end{tabular}
\caption{Job Satisfaction Summary related to degree} 
\label{tab:satisfaction}
\end{table}

\subsubsection {Career-education alignment }
\begin{itemize}
    \item \textbf{Highest Alignment}
    Health-related fields dominate with 78.9\% of graduates in closely related jobs, far surpassing other fields.
    Civil engineering follows at 70.1\%, reflecting strong industry alignment.
    \item \textbf{STEM vs. Non-STEM}     STEM fields (e.g., computer science, engineering) show moderate alignment (59-62\%), with notable exceptions like electronics/communication engineering (59.8\%).
    Non-science fields (e.g., psychology, "other non-science") still maintain >60\% alignment, suggesting broader applicability.
    \item \textbf{Lowest "Not Related" Rates}     Civil engineering (5.4\%) and health fields (5.8\%) have the fewest graduates in unrelated jobs.
    "Other non-science" fields have the highest mismatch (15.4\%).
    \item \textbf{Somewhat Related" Trends} Electronics/communication engineering has the highest percentage (28.2\%) in somewhat related roles, indicating potential skill transferability.
\end{itemize}

\begin{table}[ht]
\centering
\begin{tabular}{|l|r|r|r|r|}
  \toprule
  \hline
Field of Study & Closely Related & Not Related & Somewhat Related \\ \hline
  \midrule
  health-related fields & 78.90 & 5.80 & 10.40 \\ \hline
  civil engineering & 70.10 & 5.40 & 19.20   \\ \hline
  other non-science and engineering & 63.30 & 15.40 & 16.40  \\ \hline
  other science and engineering-related & 61.90 & 11.70 & 21.60  \\ \hline
  psychology & 61.90 & 13.80 & 19.20  \\ \hline
  computer and mathematical sciences & 61.80 & 9.40 & 23.90  \\ \hline
  electronics and comm engineering & 59.80 & 6.90 & 28.20   \\ \hline
  other physical sciences & 58.70 & 13.70 & 22.50  \\ \hline
   \bottomrule
\end{tabular}
\caption{Top 8 Fields by Percentage Closely Related to Degree} 
\label{tab:top8alignment}
\end{table}

\subsection {Analyzing work activities and educational background}
\subsubsection{distribution of work activities by degree characteristics. Higher degrees (Doctorate/Master’s) correlate with specialized roles (RD, Teaching), while Bachelor’s lean toward broader applications (Management, Tech). }
\begin{itemize}
    \item \textbf{Research \& Development (RD) Activity} \\
    Doctorate holders lead in RD involvement (60.9\%), nearly double the rate of Bachelor’s holders (34.2\%).
    Professional degree holders show the lowest RD participation (14.6\%).
    \item \textbf{Computer Application} \\
    Highest among Bachelor’s (20.6\%) and Master’s (19.0\%) graduates, likely reflecting tech-focused roles.
    Very low for Professional (2.0\%) and Doctorate (9.5\%), suggesting non-technical specializations.
    \item \textbf{Management Roles} \\
    Common across all degrees (35–53\%), peaking for Bachelor’s (53.3\%).
    Implies management is a frequent career path regardless of education level.
    \item \textbf{Teaching} \\
    Doctorate holders are most likely to teach (30.3\%), aligning with academic careers.
    Bachelor’s have the lowest teaching involvement (9.9\%).
    \item \textbf{Sample Size (n)} \\
    Bachelor’s dominate the dataset (350,788 respondents), while Professional degrees are rarest (37,811).  
\end{itemize}

\begin{table}[ht]
\centering
\begin{tabular}{|l|r|r|r|r|r|r|}
  \toprule
  \hline
Degree Type & RD(\%) & Computer Application(\%) & Management(\%) & Teaching(\%) & n \\ \hline
  \midrule
doctorate & 60.9 & 9.5 & 35.8 & 30.3 & 296088 \\ \hline
  \midrule
master's & 39.0 & 19.0 & 47.9 & 17.3 & 217423 \\ \hline
   \midrule
bachelor's & 34.2 & 20.6 & 53.3 & 9.9 & 350788 \\ \hline
   \midrule
professional & 14.6 & 2.0 & 47.6 & 15.1 & 37811 \\ \hline
   \bottomrule
\end{tabular}
\caption{Education Attainment by Working Activity} 
\label{tab:work_by_field}
\end{table}





\subsubsection{correlations between education variables and work activity engagement. Degree fields strongly predict career sector specialization—tailor education and guidance accordingly.}

\begin{itemize}
    \item \textbf{Top RD Fields} \\
    Mechanical engineering, physics, and electrical engineering lead in research (62-63\%).
    \item \textbf{Tech Roles} \\
    Electrical/electronics engineering has the highest computer application (31.5\%).
    \item \textbf{Management Focus} \\
    Civil engineering dominates management (59\%).
    \item \textbf{Teaching} \\
    Biological sciences and physics/astronomy have the most teaching roles (~20-24\%).
    \item \textbf{Sample Size} \\
    Biological sciences has the largest respondent pool (106K), reflecting strong STEM participation.
\end{itemize}


\begin{table}[H]
\centering
\begin{tabular}{|l|r|r|r|r|r|r|}
  \toprule
  \hline
Field of highest degree & RD(\%) & Computer(\%) & Managmnt(\%) & Teach(\%) & n \\ \hline
  \midrule
mechanical engineering & 62.6 & 15.4 & 46.6 & 5.5 & 37930 \\ \hline
  physics and astronomy & 62.6 & 23.9 & 29.9 & 21.6 & 29058 \\ \hline
  electrical\& com engineering & 62.0 & 31.5 & 39.2 & 6.4 & 52723 \\ \hline
  chemical engineering & 60.8 & 12.9 & 44.3 & 6.1 & 20654 \\ \hline
  chemistry, expt biochemistry & 57.6 & 7.7 & 39.7 & 17.4 & 44001 \\ \hline
  other engineering & 55.9 & 21.4 & 46.2 & 8.4 & 67774 \\ \hline
  other related sciences & 54.7 & 14.9 & 39.4 & 19.6 & 23293 \\ \hline
  biological sciences & 54.2 & 5.9 & 39.7 & 23.8 & 106104 \\ \hline
  civil engineering & 49.1 & 14.6 & 59.0 & 6.4 & 28183 \\ \hline
  other biological, agri sciences & 48.2 & 7.9 & 47.0 & 18.6 & 36078 \\ \hline
   \bottomrule
\end{tabular}
\caption{Top 10 Fields by Working Activity} 
\label{tab:work_by_field}
\end{table}





% IV
\section{Regression Analysis }
\subsection{Estimate a linear regression model predicting overall job satisfaction}

\begin{itemize}
    \item \textbf{Education Matters, But Not Always}\\
    Professional degrees boost satisfaction, but PhDs/master’s add little. Professional degrees (e.g., MBA, JD): +0.130 (highest impact). (Doctorate: +0.053 , Master’s: +0.018) Professional degree holders (e.g., lawyers, executives) report much higher satisfaction, likely due to higher pay/status.\\
    Doctorates and master’s degrees show minimal gains over bachelor’s degrees (reference group), suggesting diminishing returns for advanced education in job satisfaction.
    \item \textbf{Job-Education Relationship}\\
    Misalignment reduces satisfaction more than any other factor.     Not related (Ref): -0.336 , Somewhat related: -0.271. Jobs unrelated to education have much lower satisfaction (by 0.3+ units). Strong alignment between education and work is critical for satisfaction.
    \item \textbf{Training Participation }\\
    Employers can boost satisfaction by investing in employee development. Received training (yes): +0.077. Training is linked to moderately higher satisfaction, likely due to skill development and employer investment.
    \item \textbf{Field of Study}\\
    \:\;$\diamond$\textbf{ \:Positive Effects}: Health-related fields: +0.056. Economics: +0.060. These fields may align with stable, high-demand jobs (e.g., healthcare, finance).\\
    \:\;$\diamond$\textbf{ \:Negative Effects}: Sociology/Anthropology: -0.040. Electrical Engineering: -0.035. Potential mismatch between education and job market demands (e.g., engineering roles may involve high stress).
    \item \textbf{Demographic Gaps Exist}\\ \:\;$\diamond$\textbf{ \:Age}: Coefficient: +0.007, Each additional year of age is associated with a very slight increase in job satisfaction (0.007 units). Older workers may feel more settled or experienced in their roles.\\
    \:\;$\diamond$\textbf{ \:Gender}: Male (Ref: Female): +0.033, Men report slightly higher satisfaction than women, suggesting potential gender disparities in workplace experiences (e.g., pay, promotion opportunities, or work-life balance). \\
    \:\;$\diamond$\textbf{ \:Race/Ethnicity}: White (Ref): +0.205,(highest satisfaction) Under-represented minorities: +0.138 Other: -0.116. White workers report the highest satisfaction, while "other" racial groups (unspecified) report lower satisfaction. Under-represented minorities fall in between. This may reflect systemic inequities or cultural differences in workplace treatment.
    \item \textbf{Other Considerations}\\
    \:\;$\diamond$ Educational Attainment Paradox: Higher degrees do not guarantee higher satisfaction, possibly due to mismatched expectations or limited high-status job availability.

\end{itemize}

\begin{table}[H]
\centering
\caption{linear regression model predicting overall job satisfaction} 
\begin{tabular}{|l|r|r|r|r|r|}
  \hline
 \textbf{Function}  & \textbf{Estimate}  & \textbf{Std. Error}  & \textbf{t value}  & \textbf{Pr($>$$|$t$|$)}  \\ 
  \hline
(Intercept) & 2.4374 & 0.0103 & 237.53 & 0.0000 \\  \hline
  age & 0.0075 & 0.0002 & 46.99 & 0.0000 \\  \hline
  gender & 0.0333 & 0.0040 & 8.25 & 0.0000 \\  \hline
  racethother & -0.1164 & 0.1996 & -0.58 & 0.5598 \\  \hline
  racethunder-represented minorities & 0.1380 & 0.0064 & 21.65 & 0.0000 \\  \hline
  racethwhite & 0.2054 & 0.0053 & 39.10 & 0.0000 \\  \hline
  dgrdgdoctorate & 0.0532 & 0.0049 & 10.83 & 0.0000 \\  \hline
  dgrdgmaster's & 0.0176 & 0.0049 & 3.58 & 0.0003 \\  \hline
  dgrdgprofessional & 0.1297 & 0.0102 & 12.68 & 0.0000 \\  \hline
  ndgmedchemical engineering & 0.0020 & 0.0135 & 0.14 & 0.8852 \\  \hline
  ndgmedchemistry, except biochemistry & -0.0279 & 0.0101 & -2.75 & 0.0059 \\  \hline
  civil engineering & 0.0222 & 0.0119 & 1.86 & 0.0626 \\  \hline
  computer and mathematical sciences & -0.0263 & 0.0082 & -3.19 & 0.0014 \\  \hline
  economics & 0.0603 & 0.0119 & 5.04 & 0.0000 \\  \hline
  electrical, elect and commu engineering & -0.0349 & 0.0095 & -3.66 & 0.0003 \\  \hline
  health-related fields & 0.0560 & 0.0089 & 6.28 & 0.0000 \\  \hline
  management and administration & -0.0480 & 0.0107 & -4.50 & 0.0000 \\  \hline
  mechanical engineering & -0.0334 & 0.0107 & -3.13 & 0.0017 \\  \hline
  missing & 0.0321 & 0.7600 & 0.04 & 0.9664 \\  \hline
  other biological, agricultural, environmental & 0.0106 & 0.0114 & 0.93 & 0.3504 \\  \hline
  other engineering & 0.0099 & 0.0089 & 1.11 & 0.2669 \\  \hline
  other non-science and engineering & 0.0015 & 0.0092 & 0.17 & 0.8682 \\  \hline
  other physical and related sciences & 0.0137 & 0.0129 & 1.06 & 0.2889 \\  \hline
  other science and engineering-related & -0.0430 & 0.0130 & -3.32 & 0.0009 \\  \hline
  other social sciences & 0.0125 & 0.0121 & 1.04 & 0.3004 \\  \hline
  physics and astronomy & 0.0197 & 0.0119 & 1.66 & 0.0978 \\  \hline
  political and related sciences & 0.0286 & 0.0127 & 2.25 & 0.0246 \\  \hline
  psychology & 0.0116 & 0.0083 & 1.39 & 0.1656 \\  \hline
  sociology and anthropology & -0.0404 & 0.0111 & -3.63 & 0.0003 \\  \hline
  ocedrlpnot related & -0.3356 & 0.0057 & -58.56 & 0.0000 \\  \hline
  ocedrlpsomewhat related & -0.2712 & 0.0043 & -62.50 & 0.0000 \\  \hline
  wktrni & 0.0765 & 0.0038 & 20.24 & 0.0000 \\  \hline
   \hline
\end{tabular}
\label{tab:Job_sat_Regression}
\end{table}

\subsection{Model predicting salary satisfaction by degree}
    Regression analysis of salary satisfaction. Salary satisfaction is strongly tied to field of study, benefits, and demographic factors, with STEM careers and professional degrees offering the highest returns. Higher satisfaction is related to STEM fields, professional degrees, and strong benefits. Age, gender, and race also matter.
     
\begin{itemize}
    \item \textbf{Demographics Matter}\\
    Older employees and men report higher satisfaction, while white employees are more satisfied than underrepresented minorities.
    \item \textbf{Education Pays Off}\\
    Professional degree holders show the highest satisfaction, while doctorates report slightly lower satisfaction than bachelor’s degrees (reference group).
    \item \textbf{Field-Specific Trends}\\
    STEM fields (e.g., chemical engineering, computer science) correlate with higher satisfaction, while non-STEM fields (e.g., social sciences) often show lower satisfaction.
    \item \textbf{Benefits Impact}\\
    Satisfaction spikes for those "very satisfied" with benefits (satben), highlighting the importance of compensation packages.
    \item \textbf{Statistical Significance}\\
    Most estimates are highly significant (p < 0.05), except for master’s degrees and some fields of social science.
\end{itemize}

\begin{table}[H]
\centering
\caption{Salary Satisfaction Summary related to degree} 
\label{tab:Salary satisfaction model}
\begin{tabular}{|l|r|r|r|r|r|}
  \toprule
 Function & Estimate & Std. Error & t value & Pr($>$$|$t$|$) \\  \hline
  \midrule
(Intercept) & 1.5402 & 0.0155 & 99.56 & 0.0000 \\  \hline
  age & 0.0061 & 0.0002 & 27.19 & 0.0000 \\  \hline
  gender & 0.0484 & 0.0059 & 8.23 & 0.0000 \\  \hline
  racethunder-represented minorities & 0.0428 & 0.0087 & 4.90 & 0.0000 \\  \hline
  racethwhite & 0.1000 & 0.0074 & 13.49 & 0.0000 \\  \hline
  dgrdgdoctorate & -0.0286 & 0.0070 & -4.07 & 0.0000 \\  \hline
  dgrdgmaster's & 0.0097 & 0.0070 & 1.38 & 0.1686 \\  \hline
  dgrdgprofessional & 0.1309 & 0.0153 & 8.54 & 0.0000 \\  \hline
  ndgmedchemical engineering & 0.1866 & 0.0201 & 9.29 & 0.0000 \\  \hline
  ndgmedchemistry, except biochemistry & 0.0662 & 0.0153 & 4.32 & 0.0000 \\  \hline
  ndgmedcivil engineering & 0.0523 & 0.0179 & 2.93 & 0.0034 \\  \hline
  computer and math sciences & 0.1055 & 0.0122 & 8.67 & 0.0000 \\  \hline
  ndgmedeconomics & 0.1047 & 0.0170 & 6.16 & 0.0000 \\  \hline
  electrical, elect \& commu engineering & 0.1523 & 0.0146 & 10.41 & 0.0000 \\  \hline
  ndgmedhealth-related fields & 0.0773 & 0.0123 & 6.30 & 0.0000 \\ \hline
  management and administration & 0.1183 & 0.0157 & 7.55 & 0.0000 \\ \hline
  ndgmedmechanical engineering & 0.1383 & 0.0157 & 8.81 & 0.0000 \\ \hline
  other bio, agri, envirlife sciences & 0.0129 & 0.0176 & 0.73 & 0.4637 \\ \hline
  ndgmedother engineering & 0.1171 & 0.0130 & 9.01 & 0.0000 \\ \hline
  dother non-science and engineering & -0.0284 & 0.0135 & -2.10 & 0.0355 \\ \hline
  dother physical \& related sciences & 0.1028 & 0.0196 & 5.25 & 0.0000 \\ \hline
  dother science and engineering-related & -0.0200 & 0.0175 & -1.14 & 0.2525 \\  \hline
  ndgmedother social sciences & -0.0681 & 0.0192 & -3.54 & 0.0004 \\ \hline
  ndgmedphysics and astronomy & 0.0660 & 0.0185 & 3.58 & 0.0003 \\ \hline
  political and related sciences & -0.0033 & 0.0171 & -0.19 & 0.8471 \\ \hline
  ndgmedpsychology & -0.0134 & 0.0122 & -1.10 & 0.2708 \\ \hline
  dsociology and anthropology & -0.0520 & 0.0165 & -3.15 & 0.0016 \\ \hline
  satbensomewhat satisfied & 0.2295 & 0.0087 & 26.39 & 0.0000 \\ \hline
  satbenvery dissatisfied & 0.3810 & 0.0123 & 31.05 & 0.0000 \\ \hline
  satbenvery satisfied & 1.0953 & 0.0088 & 124.57 & 0.0000 \\ \hline
   \bottomrule
\end{tabular}
\end{table}




\subsection{Career advancement satisfaction Summary related to degree and work activity}
    The study of 178K professionals reveals that job-education alignment is the strongest predictor - satisfaction plummets by 0.34 points when work is unrelated to one's degree, while professional degree holders gain 0.19 points, underscoring the career value of specialized education.
\begin{itemize}
    \item \textbf{Education Effects}\\
    The doctorate advantage (+0.09) contrasts with master's holders' slight disadvantage (-0.02), revealing nuanced returns on graduate education that likely reflect differing career expectations and opportunities.
    \item \textbf{Notable Effects}\\
    Professional degrees (+0.19) and training participation (+0.13) boost satisfaction, while jobs unrelated to education sharply reduce it (negative 0.34).
    \item \textbf{Demographics}\\
    Women report lower satisfaction (negative 0.06) than men, and white employees score higher (+0.05) than underrepresented minorities.
    \item \textbf{Workplace Dynamics}\\
   Company size shows a clear inverse relationship - satisfaction drops 0.23-0.29 points in larger firms (500+ employees), suggesting smaller organizations may offer better advancement experiences, while training participation boosts satisfaction by 0.13 points.
    \item \textbf{Age Impact}\\
    Older employees show slightly lower satisfaction (negative 0.002 per year), though the effect is small in magnitude.
\end{itemize}





% latex table generated in R 4.4.2 by xtable 1.8-4 package
% Thu Jul 10 22:38:07 2025
\begin{table}[H]
\centering
\caption{Career advancement satisfaction Summary related to degree} 
\label{tab:career advancement satisfaction model}
\begin{tabular}{|l|r|r|r|r|r|}
  \toprule
 Function & Estimate & Std. Error & t value & Pr($>$$|$t$|$) \\ \hline
  \midrule
(Intercept) & 3.1284 & 0.0173 & 180.90 & 0.0000 \\ \hline
  age & -0.0018 & 0.0002 & -7.96 & 0.0000 \\ \hline
  Female & -0.0574 & 0.0055 & -10.46 & 0.0000 \\ \hline
  racethunder-represented minorities & -0.0126 & 0.0085 & -1.48 & 0.1379 \\ \hline
  racethwhite & 0.0481 & 0.0072 & 6.68 & 0.0000 \\ \hline
  dgrdgdoctorate & 0.0861 & 0.0070 & 12.24 & 0.0000 \\ \hline
  dgrdgmaster's & -0.0186 & 0.0067 & -2.80 & 0.0052 \\ \hline
  dgrdgprofessional & 0.1888 & 0.0142 & 13.33 & 0.0000 \\ \hline
  ocedrlpnot related & -0.3387 & 0.0086 & -39.54 & 0.0000 \\ \hline
  ocedrlpsomewhat related & -0.1887 & 0.0063 & -29.88 & 0.0000 \\ \hline
  wktrni & 0.1256 & 0.0055 & 23.00 & 0.0000 \\ \hline
  waprsmmanagement and administration & 0.0662 & 0.0113 & 5.85 & 0.0000 \\ \hline
  waprsmother & -0.0719 & 0.0118 & -6.09 & 0.0000 \\ \hline
  waprsmresearch and development & -0.0128 & 0.0114 & -1.13 & 0.2596 \\ \hline
  waprsmteaching & -0.0890 & 0.0128 & -6.96 & 0.0000 \\ \hline
  emsize100-499 employees & -0.2878 & 0.0110 & -26.18 & 0.0000 \\ \hline
  emsize1000-4999 employees & -0.2637 & 0.0107 & -24.68 & 0.0000 \\ \hline
  emsize11-24 employees & -0.1802 & 0.0152 & -11.85 & 0.0000 \\ \hline
  emsize25-99 employees & -0.2781 & 0.0122 & -22.75 & 0.0000 \\ \hline
  emsize25000+ employees & -0.2296 & 0.0095 & -24.15 & 0.0000 \\ \hline
  emsize500-999 employees & -0.2880 & 0.0131 & -22.06 & 0.0000 \\ \hline
  emsize5000-24999 employees & -0.2349 & 0.0106 & -22.17 & 0.0000 \\ \hline
   \bottomrule
\end{tabular}
\end{table}



\section{Data Visualization}
\subsection{ the relationship between salary and overall job satisfaction}

\begin{itemize}
    \item \textbf{Salary-Satisfaction Relationship}
    The positive slopes of regression lines suggest higher salaries generally correlate with higher job satisfaction across all degree levels.The strength of this relationship varies by degree type (steeper slopes for Doctorate/Professional degrees).
    \item \textbf{Degree-Level Differences}
    \item Professional degree holders (red line) likely show highest satisfaction at comparable salaries strongest salary-satisfaction relationship (steepest slope). Bachelor's degrees (blue) may show Flatter slope, suggesting salary increases matter less for their satisfaction lower satisfaction at higher salaries compared to advanced degrees.
    \item \textbf{Potential Insights}
    The "Professional" group's curve suggests their satisfaction is more sensitive to salary changes than other groups. Master's and Doctorate lines appear closer, possibly indicating similar satisfaction drivers. At lower salaries (<\$50K), satisfaction differences between degrees are minimal, but they diverge sharply at higher salaries.
\end{itemize}


\begin{figure}[h]
\centering
\includegraphics[width=0.8\textwidth]{Salary_job_satisfaction.png}
\caption{Relationship between salary and overall job satisfaction}
\end{figure}


\end{document}